\Introduction

\textbf{Актуальность темы.} Компьютерное обучение языкам охватывает широкий спектр приложений и подходов к информационно-коммуникационным технологиям для преподавания и изучения иностранных языков, начиная от традиционных методов обучения, которые зародились в 1960-х и 1970-х годах, и заканчивая более современными подходами с использованием виртуальной учебной среды и дистанционного обучения с помощью Интернета. На сегодняшний день компьютерное обучение иностранным языкам является одним из бурно развивающихся направлений.

Современные методы обучения требуют активного использования компьютерных технологий, которые позволяют усовершенствовать процесс обучения не только преподавателям, но и обучающимся. Нынешняя концепция компьютерного обучения по большей части направлена на самостоятельную работу учащихся: материалы разработаны таким образом, что сочетают в себе различные интерактивные элементы и индивидуальное обучение, что способствует развитию умений и навыков, необходимых учащимся в процессе обучения и самообразования. Однако эффективность данного типа обучения достигается в совокупности с традиционными методами и помогает учителям облегчить процесс обучения языку: к примеру, было изучено влияние использования одной из наиболее популярных онлайн-платформ по изучению иностранных языков~--- Duolingo, внедрив ее в класс студентов, изучающих испанский. Как показывают результаты данного исследования, те студенты, которые на регулярной основе выполняли задания в Duolingo, демонстрировали лучшие результаты на тестах по сравнению с теми студентами, которые не выполняли этих упражнений \cite{sushant}.

\textbf{Объектом исследования} выступают обучающие методы и интерактивные подходы, используемые в современных онлайн-платформах для обучения иностранному языку.

\textbf{Предметом исследования} является моделирование алгоритмов работы обучающих методов для их последующего внедрения в разрабатываемое веб-приложение.

\textbf{Целью исследования} является разработка реактивного веб-приложения для изучения иностранного языка. Под \textit{реактивностью} подразумевается способность приложения отслеживать изменения в своем состоянии, распространять сведения об этих изменениях другим компонентам и вовремя отрисовывать представление данных, а также своевременно реагировать на действия пользователя, что играет немаловажную роль для достижения интерактивности.

Для достижения поставленной цели были сформулированы следующие \textbf{задачи}:

\begin{itemize}
	\item провести анализ наиболее эффективных для обучения педагогических технологий и смоделировать алгоритмы их работы;
	\item изучить документации к инструментам разработки;
	\item осуществить проектирование и разработку приложения.
\end{itemize}

\textbf{Средствами разработки} в данной работе выступают: язык для описания структурной разметки HTML, язык для стилизации элементов CSS, язык программирования JavaScript, веб-фреймворк Vue.js, а также JavaScript-библиотеки с открытым исходным кодом: Axios, Vue Material, Vuelidate, Vuex и Vue Router. В роли серверной составляющей для организации обмена и хранения данных служит облачная нереляционная база данных Cloud Firestore, входящая в состав сервисов Firebase от компании Google.

%%% Local Variables:
%%% mode: latex
%%% TeX-master: t
%%% End: