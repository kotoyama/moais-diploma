\Abbreviations %% Список обозначений и сокращений в тексте
\begin{description}
%\item[SQL] Structured Query Language, <<язык структурированных запросов>>. %Применяется в реляционной модели для управления данными.
\item[NoSQL] Not only Structured Query Language, <<не только SQL>>. Подход, основанный на нереляционной модели базы данных.
\item[API] Application Programming Interface, <<программный интерфейс приложения>>. Посредник между теми или иными сервисами для обеспечения доступа и обмена данными.
\item[HTML] HyperText Markup Language, <<язык гипертекстовой разметки>>. Описывает разметку и применяется для структурирования и представления содержимого веб-страницы.
\item[CSS] Cascading Style Sheets, <<каскадные таблицы стилей>>. Используется для стилизации и общего оформления внешнего вида веб-страницы.
\item[DOM] Document Object Model, <<объектная модель документа>>. Программный интерфейс, который выстраивает иерархическую структуру HTML-документов.
\item[JSON] JavaScript Object Notation, <<текстовый формат обмена данными, основанный на JavaScript>>. Описывает структуру и организацию данных и предоставляет возможность их хранения и передачи.
%\item[MVVM] Model~--- View~--- ViewModel, <<модель~--- представление~--- модель представления>>. Шаблон проектрирования архитектуры приложений, позволяющий создавать приложения, которые мгновенно реагируют на действия пользователя.
\item[NPM] Node Package Manager, <<менеджер пакетов>>. Инструмент для управления программными JavaScript-пакетами, который позволяет производить их установку одной командой.
\end{description}

%%% Local Variables:
%%% mode: latex
%%% TeX-master: "rpz"
%%% End:
